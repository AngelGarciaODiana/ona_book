\usepackage{booktabs}
\usepackage{longtable}
\usepackage{graphicx}
\usepackage[bf,singlelinecheck=off]{caption}
\usepackage[scale=.8]{sourcecodepro}


\usepackage{framed,color}
\definecolor{shadecolor}{RGB}{248,248,248}

\renewcommand{\textfraction}{0.05}
\renewcommand{\topfraction}{0.8}
\renewcommand{\bottomfraction}{0.8}
\renewcommand{\floatpagefraction}{0.75}

\renewenvironment{quote}{\begin{VF}}{\end{VF}}
\let\oldhref\href
\renewcommand{\href}[2]{#2\footnote{\url{#1}}}

\makeatletter


\usepackage{makeidx}
\makeindex

\urlstyle{tt}

\usepackage{amsthm}
\makeatletter
\def\thm@space@setup{%
  \thm@preskip=8pt plus 2pt minus 4pt
  \thm@postskip=\thm@preskip
}
\makeatother

\usepackage[nonumberlist]{glossaries}
\makenoidxglossaries

\newglossaryentry{graph}
{
    name={graph},
    description={A mathematical structure consisting of a set of vertices and a separate set of pairs of vertices, known as edges}
}

\newglossaryentry{network}
{
    name={network},
    description={A group or system of interconnected entities (people or things), usually modeled using a graph}
}

\newglossaryentry{graphdatabase}
{
    name={graph database},
    description={A persistent data storage system that is structured similar to a graph, with a vertex set and an edge set}
}

\newglossaryentry{vertex}
{
    name={vertex},
    description={An element of the vertex set of a graph, representing an entity (person or thing) which may or may not be connected to other entities; also known as a node}
}

\newglossaryentry{node}
{
    name={node},
    description={An element of the vertex set of a graph, representing an entity (person or thing) which may or may not be connected to other entities; also known as a vertex}
}

\newglossaryentry{edge}
{
    name={edge},
    description={An element of the edge set of a graph, representing a connection between two vertices}
}

\newglossaryentry{degreecentrality}
{
    name={degree centrality},
    description={The degree centrality (or simply the degree) of a vertex is the number of edges connected to it in the graph; it is a common measure of the vertex's importance}
}

\newglossaryentry{betweennesscentrality}
{
    name={betweenness centrality},
    description={The betweenness centrality of a vertex is the proportion of times it lies on the shortest path between all other pairs of vertices in the graph; it is a measure of the role of the vertex in the overall connectedness of the graph}
}

\newglossaryentry{closenesscentrality}
{
    name={closeness centrality},
    description={The closeness centrality of a vertex is the inverse of the sum of the distances from that vertex to all other vertices in the graph; it is a measure of how efficiently other vertices can be reached from that vertex}
}

\newglossaryentry{eigenvectorcentrality}
{
    name={eigenvector centrality},
    description={The eigenvector centrality (or eigencentrality) of a vertex is calculated using the unique largest eigenvalue  and corresponding eigenvector of the adjacency matrix of the graph; it is a measure of the influence of a vertex in the sense that it is connected to other vertices that are important in the network.  It is also known as relative centrality or prestige}
}

\newglossaryentry{walk}
{
    name={walk},
    description={A route from one vertex to another using edges in the graph, also known as a path}
}

\newglossaryentry{path}
{
    name={path},
    description={A route from one vertex to another using edges in the graph, also known as a walk}
}

\newglossaryentry{temporalnetwork}
{
    name={temporal network},
    description={A network whose characteristics can change over time}
}

\newglossaryentry{distance}
{
    name={distance},
    description={The length of the shortest path between two vertices in a graph}
}

\newglossaryentry{pathlength}
{
    name={path length},
    description={The sum of the weights of the edges on a path between two vertices in a graph.  If an unweighted graph, then the edge weights are taken as 1}
}

\newglossaryentry{diameter}
{
    name={diameter},
    description={The largest distance between any pair of vertices in a graph}
}

\newglossaryentry{communitydetection}
{
    name={community detection},
    description={An algorithmic process to partition a graph into subsets of vertices in order to maximize the connectedness within subsets and minimize the connectedness between subsets.  Common algorithms for community detection include the Louvain, Leiden and Girvan-Newman algorithms}
}

\newglossaryentry{preferentialattachment}
{
    name={preferential attachment},
    description={A phenomenon observed in some real world networks where vertices are more likely to connect to high degree (or popular) vertices}
}

\newglossaryentry{scalefreenetworks}
{
    name={scale-free networks},
    description={A network where the degrees of the vertices obey a power law distribution with certain parameters; such networks exist rarely but are of interest because they can be the result of rapid network growth with preferential attachment}
}

\newglossaryentry{adjacentvertices}
{
    name={adjacent vertices},
    description={Two vertices which are connected by an edge in a graph}
}

\newglossaryentry{simplegraph}
{
    name={simple graph},
    description={A graph where there is no more than one edge between any two vertices and where there are no loop edges}
}

\newglossaryentry{loopedge}
{
    name={loop edge},
    description={An edge that starts and ends on the same vertex}
}

\newglossaryentry{multigraph}
{
    name={multigraph},
    description={A graph where more than one edge can exist between any pair of vertices}
}

\newglossaryentry{pseudograph}
{
    name={pseudograph},
    description={A graph where loop edges are allowed}
}

\newglossaryentry{completegraph}
{
    name={complete graph},
    description={A graph where all pairs of vertices are connected; otherwise stated, a graph with an edge density of 1}
}

\newglossaryentry{clique}
{
    name={clique},
    description={A subset of vertices in a graph whose induced subgraph is a complete graph}
}

\newglossaryentry{inducedsubgraph}
{
    name={induced subgraph},
    description={A graph formed from a vertex subset of a larger graph and all edges connecting the vertices in the vertex subset, also called a vertex subgraph or simply subgraph}
}

\newglossaryentry{edgesubgraph}
{
    name={edge subgraph},
    description={A graph formed from an edge subset of a larger graph and the vertices connected by that edge subset}
}

\newglossaryentry{bipartitegraph}
{
    name={bipartite graph},
    description={A graph consisting of two distinct connected subgraphs}
}

\newglossaryentry{connectedgraph}
{
    name={connected graph},
    description={A graph where a path exists between all pairs of vertices}
}

\newglossaryentry{kpartitegraph}
{
    name={k-partite graph},
    description={A graph consisting of $k$ distinct connected subgraphs}
}

\newglossaryentry{tree}
{
    name={tree},
    description={A connected graph where there is only one path between any pair of vertices; when visualized, such a graph has a tree structure}
}

\newglossaryentry{property}
{
    name={vertex or edge property},
    description={Information that is attached to or stored in the vertices or edges of a graph}
}

\newglossaryentry{edgeweight}
{
    name={edge weight},
    description={A common name for a numeric edge property}
}

\newglossaryentry{centrality}
{
    name={centrality},
    description={A general term for the importance of a vertex in the connective structure of a graph; can be defined in a variety of ways}
}

\newglossaryentry{community}
{
    name={community},
    description={A subset of vertices in a graph which are considered to have a relatively dense connected structure}
}

\newglossaryentry{edgelist}
{
    name={edgelist},
    description={A representation of a graph as a list of edges (pairs of vertices in the edge set), often with other information such as edge weight}
}

\newglossaryentry{isolate}
{
    name={isolate},
    description={A vertex which is not connected to any other vertices in a graph; also known as a singleton}
}

\newglossaryentry{adjacencymatrix}
{
    name={adjacency matrix},
    description={A representation of a graph as a square matrix indexed by the vertices, where the $(i,j)$-th entry is the weight of the edge from vertex $i$ to vertex $j$ or zero if such an edge does not exist; weights are considered to be 1 for unweighted graphs}
}

\newglossaryentry{layout}
{
    name={layout},
    description={A positioning of the vertices in a graph visualization, usually calculated by means of a selected algorithm}
}

\newglossaryentry{randomseed}
{
    name={random seed},
    description={A value used in programming to control and replicate random number generation.  In graph layouts, this is used to ensure that a layout can be reproduced by others}
}

\newglossaryentry{forcedirected}
{
    name={force-directed layout},
    description={A popular and aesthetically pleasing layout of vertices which is calculated by means of physical simulation algorithms.  Force-directed algorithms are good at finding layouts where the edges are of similar length and where as few edges cross as possible}
}

\newglossaryentry{hairball}
{
    name={hairball},
    description={A common phenomenon in the visualization of complex graphs, where a group of densely connected vertices resemble a ball or clump of hair}
}

\newglossaryentry{density}
{
    name={density (edge density)},
    description={The number of edges in a graph expressed as a proportion of the total possible number of edges in the graph, most commonly used in the description of simple graphs.  A higher density graph means greater likelihood of connection between any pair of its vertices}
}

\newglossaryentry{sparse}
{
    name={sparse graph},
    description={A graph considered to have low edge density}
}

\newglossaryentry{scraping}
{
    name={scraping},
    description={A process whereby elements of text are extracted from electronic documents}
}

\newglossaryentry{regex}
{
    name={regex},
    description={Regular expression syntax, a common syntax used to search text for certain patterns or strings}
}

\newglossaryentry{graphtraversal}
{
    name={graph traversal},
    description={An algorithmic process of visiting, checking and updating vertices in a graph, commonly used in graph search or shortest path problems}
}

\newglossaryentry{simplepath}
{
    name={simple path (or acyclic path)},
    description={A path between two vertices in a graph where no vertex is repeated}
}

\newglossaryentry{breadthfirstsearch}
{
    name={breadth-first search},
    description={A graph traversal method which moves through all neighboring vertices of a given vertex before moving to neighbors of neighbors}
}

\newglossaryentry{depthfirstsearch}
{
    name={depth-first search},
    description={A graph traversal method which selects a neigboring vertex of a given vertex and moves as far down the resulting path as possible before returning to the starting point and visiting another neighboring vertex}
}

\newglossaryentry{shortestpathalgorithm}
{
    name={shortest path algorithms},
    description={A class of graph traversal algorithms designed to find the shortest length paths between vertices, including Dijkstra, Bellman-Ford, Johnson and Floyd-Warshall}
}

\newglossaryentry{egonetwork}
{
    name={ego network},
    description={The $n$-th order ego network of a vertex is a set consisting of that vertex plus all vertices that are at distance at most $n$ from it}
}

\newglossaryentry{egosize}
{
    name={ego size},
    description={The $n$-th order ego size of a vertex is the number of vertices in its $n$-th order ego network}
}

\newglossaryentry{hubscore}
{
    name={hub score},
    description={The hub score of a vertex in a directed graph is the outgoing eigenvector centrality of that vertex}
}

\newglossaryentry{authorityscore}
{
    name={authority score},
    description={The authority score of a vertex in a directed graph is the incoming eigenvector centrality of that vertex}
}

\newglossaryentry{clustering}
{
    name={clustering},
    description={The process of partitioning a graph into subsets of vertices with high density, strongly associated with community detection}
}

\newglossaryentry{partition}
{
    name={partition},
    description={A division of the vertices of a graph into mutually exclusive subsets}
}

\newglossaryentry{weaklyconnected}
{
    name={weakly connected},
    description={A directed graph is said to be weakly connected if it is connected when viewed as an undirected graph}
}

\newglossaryentry{stronglyconnected}
{
    name={strongly connected},
    description={A directed graph is said to be strongly connected if there is a directed path between all pairs of vertices in the graph}
}

\newglossaryentry{unilaterallyconnected}
{
    name={unilaterally connected},
    description={A directed graph is said to be unilaterally connected there is a path in any direction between all pairs of vertices in the graph}
}

\newglossaryentry{connectedcomponent}
{
    name={connected component},
    description={A subset of vertices in a graph whose induced subgraph is connected}
}

\newglossaryentry{cut}
{
    name={cut},
    description={A set of edges in a connected graph whose removal would split the graph into distinct connected components}
}

\newglossaryentry{minimumcut}
{
    name={minimum cut},
    description={A cut for which no other cut exists with fewer edges}
}

\newglossaryentry{modularity}
{
    name={modularity},
    description={For a partition of a graph, the modularity measures the extent to which in group density is greater than what would be expected by chance.  Modularity is an important measure in community detection as high modularity indicates a strong community structure}
}

\newglossaryentry{maximalclique}
{
    name={maximal clique},
    description={A clique where the addition of any other vertex would render it no longer a clique}
}

\newglossaryentry{assortativity}
{
    name={assortativity},
    description={The extent to which vertices with a given property are more likely to connect to vertices with the same property}
}

\newglossaryentry{nominalassortativity}
{
    name={nominal assortativity},
    description={Assortativity based on a categorical property of vertices}
}

\newglossaryentry{degreeassortativity}
{
    name={degree assortativity},
    description={Assortativity based on the degree centrality of vertices}
}

\newglossaryentry{vertexsimilarity}
{
    name={vertex similarity},
    description={The extent to which two vertices have similar neighboring vertices.  Measured using Jaccard, dice or log-weighted similarity coefficients}
}

\newglossaryentry{graphsimilarity}
{
    name={graph similarity},
    description={The extent to which two graphs have a similar structure.  In simpler situations this reduces to the similarity of the edge sets of two graphs with the same vertex sets}
}

\newglossaryentry{labeledpropertygraph}
{
    name={labeled-property graph},
    description={A form of graph database where the nodes and edges contain defined properties.  These tend to be easier to query but are less flexible in handling rapidly changing data structures}
}

\newglossaryentry{rdf}
{
    name={resource description framework (RDF)},
    description={A form of graph database which allows new entities and relationships be 'spawned' flexibly.  RDFs are more flexible in handling rapidly changing data structures but querying is more complex}
}

\newglossaryentry{cypher}
{
    name={Cypher query language},
    description={An increasingly popular query language developed for Neo4J graph databases which uses ASCII art to denote nodes and edges}
}

\newglossaryentry{sparql}
{
    name={SPARQL},
    description={The standard query language for resource description framework (RDF) graph databases}
}



\glsaddall

\usepackage{tcolorbox}

\definecolor{babyblueeyes}{rgb}{0.63, 0.79, 0.95}
\definecolor{fullblue}{rgb}{0, 0, 1}

\newtcolorbox{thinkahead}{
  colback=babyblueeyes,
  colframe=fullblue,
  boxsep=5pt,
  arc=4pt}
  
\usepackage{footnote}
\usepackage{etoolbox}
\BeforeBeginEnvironment{thinkahead}{\savenotes}
\AfterEndEnvironment{thinkahead}{\spewnotes}

\frontmatter
